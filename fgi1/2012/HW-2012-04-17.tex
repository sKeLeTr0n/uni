\documentclass[a4paper]{scrartcl}
\usepackage[ngerman]{babel}
\usepackage[utf8]{inputenc}
\usepackage[T1]{fontenc}
\usepackage{lmodern}
\usepackage{amssymb}
\usepackage{amsmath}
\usepackage{enumerate}
\usepackage{scrpage2}\pagestyle{scrheadings}
\usepackage{tikz}
\usetikzlibrary{arrows,automata}

\newcommand{\titleinfo}{Hausaufgaben zum 12.04.2012}
\newcommand{\aufgabe}[1]{\item[\textbf{#1}]}

\title{\titleinfo}
\author{Arne Feil}
\date{\today}
\ihead{FGI1}
\chead{Arne Feil, Sven-Hendrik Haase, Christian Darsow-Fromm}
\ohead{\today}
\setheadsepline{1pt}
\newcommand{\qed}{\quad \square}
\everymath{\displaystyle}
\begin{document}
%\maketitle
\begin{enumerate}
\aufgabe{2.3}
NFA:\\
\begin{tikzpicture}[->, auto, node distance=3cm, >=latex]
    \tikzstyle{every initial by arrow}=[initial text=,->, >=stealth]

    \node[initial, state] (A) {$q_1$};
    \node[accepting, state] (B) [right of=A] {$q_2$};
    \node[accepting, state] (C) [right of=B] {$q_3$};

    \draw (A) edge [bend left] node {b} (B);
    \draw (B) edge [loop above] node {a,b,d} (B);
    \draw (B) edge [bend left] node {c} (C);
    \draw (C) edge [loop above] node {c} (C);
    \draw (C) edge [bend left] node {d} (B);
\end{tikzpicture}
\aufgabe{2.4}
$w \in \{0,1\}* \quad L=\Big\{w\overline{w} | w \in \{0,1\}* \Big\} \\
w = w^n \overline{w}^n \quad \quad |w| \geq n \\
w = uvw' \quad \quad |uv| \leq n \\
w = w^m \quad \quad m \in \mathbb{N} \quad \quad 1 \leq m \leq n \\
w^{n-m}w^m\overline{w}^n \in L \quad m>0 \\
w^{n-m} w^m w^m \overline{w}^n = w^{n+m}\overline{w}^m \notin L$

\aufgabe{2.5}
\begin{enumerate}[1.]
\item
% Wozu ein Beispiel?
Beispiel: \\
\(\Sigma = \{0, 1\}\) \\
Sei \(a = 1\). \\
Einige Worte aus \(L: w_1 = 11111, w_2 = 11011, w_3 = 00101, w_4 = 1, w_5 = 10\) \\
Einige Worte aus \((L\%a)\) mit der beschriebenen Kondition\(: v_1 = 1111, v_2 = 11101, v_3 = 0010, v_4 = \{\}, v_5 = \{\}\) \\
\\
\(L \subseteq \Sigma^*\) ist eine reguläre Sprache, weil alle ihre Worte mit
dem regulären Ausdruck \(\{\Sigma^*\}\) beschrieben werden können. \(L\%a\) ist
eine reguläre Sprache, weil alle ihre Worte mit dem regulären Ausdruck (0|1)*(?=1)
beschrieben werden können. Außerdem gilt \((L\%a) \subseteq L \).\\
Allgemein ist also der reguläre Ausdruck für \((L\%a)\): \(\Sigma^*(?=a)\).

\item
$(KW) \subseteq L$ \\
Da $L$ regulär ist, ist es eine endliche Teilmenge von $\Sigma^*$. $KW$ ist als
Teilmenge von $L$ auch endlich und somit auch eine reguläre Sprache.
\end{enumerate}
\end{enumerate}
\end{document}
