\documentclass[a4paper]{scrartcl}
\usepackage[ngerman]{babel}
\usepackage[utf8]{inputenc}
\usepackage[T1]{fontenc}
\usepackage{lmodern}
\usepackage{amssymb}
\usepackage{amsmath}
\usepackage{enumerate}
\usepackage{scrpage2}\pagestyle{scrheadings}
\usepackage{tikz}
\usetikzlibrary{arrows,automata}

\newcommand{\titleinfo}{Hausaufgaben zum 01.05.2012}
\newcommand{\aufgabe}[1]{\item[\textbf{#1}]}

\title{\titleinfo}
\author{Arne Feil}
\date{\today}
\ihead{FGI1}
\chead{Arne Feil, Sven-Hendrik Haase, Christian Darsow-Fromm}
\ohead{\today}
\setheadsepline{1pt}
\newcommand{\qed}{\quad \square}
\newcommand{\terminal}[1]{\langle #1 \rangle }
\everymath{\displaystyle}
\begin{document}
%\maketitle
\begin{enumerate}

\aufgabe{4.3}
\begin{enumerate}[1.]
 \item
 $X \in \{A,B,\bot\}$\\
 $X$ steht für ein beliebiges der drei möglichen Zeichen im Keller.\\
 \begin{tikzpicture}[->, auto, node distance=3cm, >=latex]
    \tikzstyle{every initial by arrow}=[initial text=,->, >=stealth]

    \node[initial, state] (A) {$q_1$};
    \node[state] (B) [right of=A] {$q_2$};
    \node[accepting, state] (C) [right of=B] {$q_3$};

    \draw (A) edge [loop above] node[text width=2cm,align=center] {$a,X|AX$\\
									  $b,X|BX$} (A);
    \draw (A) edge [bend left] node[text width=2cm,align=center, below] {$a,X|AX$\\
									  $b,X|BX$\\
									  $a,X|X$\\ % ändert nichts am Keller.
									  $b,X|X$} (B);
    \draw (B) edge [loop above] node[text width=2cm,align=center] {$a,A|\epsilon$\\
									  $b,B|\epsilon$} (B);
    \draw (B) edge [bend left] node[text width=2cm,align=center, below] {$\epsilon,\bot|\epsilon$} (C);
\end{tikzpicture}

Im Zustand $q_1$ wird jeder Buchstabe in den Keller geschrieben.
Der Wechsel zu $q_2$ kann mit jedem Kellerinhalt passieren, entweder mit dem letzten Buchstaben der ersten Worthälfte oder mit dem mittleren Buchstaben.
Bei einer geraden Buchstabenzahl wird er bei dem Übergang in den Keller geschrieben, bei einer ungeraden nicht.
Die Schleife bei $q_2$ läuft so lange, bis der Keller leer ist. Wenn er zu früh leer ist, erreicht er keinen Endzustand.
Das Schlusszeichen $\bot$ wird mit dem letzten $\epsilon$-Übergang zu $q_2$ aus dem Keller geholt. Wenn das alles erfolgreich war, erreicht der Automat dort seinen Endzustand.

\item
\begin{align*}
 L &= \{a^n b^* c^n | b\text{ kann nur folgen, wenn } n\geq 1\};\ n\in\mathbb{N};\ n\geq 0\\
 G &= (\Sigma,N,P,S)\\
 \Sigma &= \{a,b,c\}\\ %Muss hier auch das \epsilon auftauchen?
 N &= \{S,T\}\\
 P &= \{
   S\to \epsilon,\
   S\to aSc,\
   S\to aTc,\
   T\to bT,\
   T\to \epsilon
 \}\\
 S &= S
\end{align*}

\paragraph{Beweis von $L(G)=L$}

\subparagraph{$L(G)\subseteq L$}

Das leere Wort ist in $L(G)$ enthalten. Da die Sprache $L$ mit $\{\dots\}^*$ definiert ist, enthält sie auch $\epsilon$. \\
Die weiteren Möglichkeiten von $S$ sind:

\begin{description}
\item[$S\to aSc$]\
gleiche Anzahl von $a$ und $c$ mit noch Unbekanntem in der Mitte. Die bisherigen Terminale sind in $L$ enthalten.

\item[$S\to aTc$]\
es kommt ein neues $a$ an den Anfangsblock und ein $c$ an den Endblock. Auch hier sind die Terminale in $L$.
\end{description}

Nun geht es mit $T$ weiter. $T\to bT$ erzeugt eine beliebig lange Reihe $\{b\}^*$ und bricht mit $T\to \epsilon$ ab.
Die Reihe von $b$ innerhalb der $a$ und $c$ ist auch in der Sprache enthalten. Damit stimmt die Aussage $L(G)\subseteq L$.

\subparagraph{$L\subseteq L(G)$}

In $L$ sind am Anfang beliebig viele ($n$) $a$ und am Ende genauso viele $c$.
Für $n=0$ gibt es $S\to \epsilon$.

Die $a$ und $c$ lassen sich mit $S\to aSc$ erzeugen und sind damit in $L(G)$.
Nach dem Wechsel über $S\to aTc$ lassen sich mit $T\to bT$ beliebig viele $b$ erzeugen.
Diese sind in der Mitte des Wortes und damit an der richtigen Position.
$T\to \epsilon$ sorgt dafür, dass nicht unbedingt ein $b$ existieren muss und die Schleife beendet werden kann.

Damit sind alle erforderlichen Bedingungen von $L(G)$ erfüllt und es gilt $L=L(G)$.

\end{enumerate}


\aufgabe{4.4}

% Chomsky-Normalform.
\begin{enumerate}[1.]
\item
$\epsilon$-frei machen \\
$
S \rightarrow BCD~|~BC~|~BD~|~B~|~a \\
A \rightarrow aBD~|~aDb~|~CSSD~|~aB~|~ab~|~CSS~|~SSD \\
B \rightarrow bB~|~aC~|~bD~|~a~|~b \\
C \rightarrow Cc~|~c \\
D \rightarrow AC~|~A
$

\item
Reduzieren \\
Es sind keine unproduktiven Terminale enthalten.

\item
Kettenregeln entfernen \\
$
S \rightarrow BCD~|~BC~|~BD~|~bB~|~aC~|~bD~|~a~|~b \\
A \rightarrow aBD~|~aDb~|~CSSD~|~aB~|~ab~|~CSS~|~SSD \\
B \rightarrow bB~|~aC~|~bD~|~a~|~b \\
C \rightarrow Cc~|~c \\
D \rightarrow AC~|~aBD~|~aDb~|~CSSD~|~aB~|~ab~|~CSS~|~SSD
$

\item
Ersetzen langer Terminalregeln \\
$
S \rightarrow BCD~|~BC~|~BD~|~\terminal{b}B~|~\terminal{a}C~|~\terminal{b}D~|~a~|~b \\
A \rightarrow \terminal{a}BD~|~\terminal{a}D\terminal{b}~|~CSSD~|~\terminal{a}B~|~\terminal{a}\terminal{b}~|~CSS~|~SSD \\
B \rightarrow \terminal{b}B~|~\terminal{a}C~|~\terminal{b}D~|~a~|~b \\
C \rightarrow C\terminal{c}~|~c \\
D \rightarrow AC~|~\terminal{a}BD~|~\terminal{a}D\terminal{b}~|~CSSD~|~\terminal{a}B~|~\terminal{a}\terminal{b}~|~CSS~|~SSD \\
\terminal{a} \rightarrow a \\
\terminal{b} \rightarrow b \\
\terminal{c} \rightarrow c
$

\item
Verkürzen zu langer Regeln \\
$
S \rightarrow \terminal{BC}D~|~BC~|~BD~|~\terminal{b}B~|~\terminal{a}C~|~\terminal{b}D~|~a~|~b \\
A \rightarrow \terminal{a}\terminal{BD}~|~\terminal{\terminal{a}D}\terminal{b}~|~\terminal{CS}\terminal{SD}~|~\terminal{a}B~|~\terminal{a}\terminal{b}~|~\terminal{CS}S~|~S\terminal{SD} \\
B \rightarrow \terminal{b}B~|~\terminal{a}C~|~\terminal{b}D~|~a~|~b \\
C \rightarrow C\terminal{c}~|~c \\
D \rightarrow AC~|~\terminal{a}\terminal{BD}~|~\terminal{\terminal{a}D}\terminal{b}~|~CSSD~|~\terminal{a}B~|~\terminal{a}\terminal{b}~|~\terminal{CS}S~|~S\terminal{SD} \\
\terminal{a} \rightarrow a \\
\terminal{b} \rightarrow b \\
\terminal{c} \rightarrow c \\
\terminal{BC} \rightarrow BC \\
\terminal{BD} \rightarrow BD \\
\terminal{\terminal{a}D} \rightarrow \terminal{a}D \\
\terminal{CS} \rightarrow CS \\
\terminal{SD} \rightarrow SD
$

\item
% Ob die Begründung formuliert wie es erwartet wird, weiß ich nicht, sie stimmt aber ;)
Wiederherstellen der ursprünglichen Sprache durch evtl. Hinzunahme einer $\epsilon$-Regel \\
Es muss keine $\epsilon$-Regel hinzugefügt werden, da in der Sprache keine leeres Wort erzeugt werden konnte. \\
$
S \rightarrow BCD $ mit $C \rightarrow \epsilon$, $D \rightarrow \epsilon$ und $B \rightarrow bB~|aC~|~bD$ \\
$S$ wird also mindestens auf $a$, oder $b$ abgeleitet.
\end{enumerate}


\aufgabe{4.5}

% TODO Bonusaufgabe.

\end{enumerate}
\end{document}
