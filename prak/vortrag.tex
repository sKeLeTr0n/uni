\documentclass{beamer}
\usepackage[ngerman]{babel}
\usepackage[utf8x]{inputenc}
\usepackage{pgfplots}
\usepackage{tikz}
\usepackage{amsmath,amsfonts,amssymb}
\pdfmapfile{+sansmathaccent.map}
\usetheme{Warsaw}
\usepackage{graphicx}
\begin{document}

\title{ParaQuest}
\subtitle{Hochleistungsspeicherzugriffsfehler}
\date{2013-09-23}
\author{Alisa Dammer \and Sven-Hendrik Haase}

\begin{frame}
    \titlepage
\end{frame}

\section*{Gliederung}
\begin{frame}{Gliederung}
    \tableofcontents
\end{frame}

\section{Problemstellung}
\begin{frame}{Problemstellung}
\begin{itemize}
    \item Grid-basierte 2D Welt mit Kreaturen, die gegeneinander kämpfen
    \item Einfaches Kampfsystem mit Trefferpunkten, Angriffsstärke und Ausweichchance
    \item Verschiedene Spezies von Kreaturen: Goblin, Hobbit, Orc, Elf, Dwarf, Human
    \item Jede Spezies hat andere Werte der Attribute Strength und Agility
    \item 1 Schritt pro Tick pro Kreatur
    \item Zusätzlich gesperrte Felder (hier Steine)
    \item Schicke Visualisierung der Welt
\end{itemize}
\end{frame}


\section{Lösungsansatz}
\begin{frame}{Lösungsansatz}
\begin{itemize}
    \item Welt als sparse Matrix
    \item Jeden Tick ein Dump der gesamten Welt dieses Prozesses
    \item Jede Kreatur hat eine 2D Position und eine global eindeutige Id
    \item Umgesetzt in C++11 mit Visualisierung in Qt sowie cereal für die Serialisierung
    \item Folgender Ablauf pro Tick: Kreaturen bewegen, Prozesse synchronisieren, Kämpfen, Verlierer löschen, Welt Dump
    \item Der Welt Dump wird später für die Visualisierung benötigt
\end{itemize}
\end{frame}


\section{Parallelisierungsschema}
\begin{frame}{Parallelisierungsschema}
\begin{itemize}
	\item Kreaturen werden am Anfang auf Hauptprozess erzeugt und auf andere Prozesse aufgeteilt
    \item Die Welt wird am Anfang der Simulation spaltenweise aufgeteilt
    \item Die Aufteilung ist statisch
    \item Es wird für Kollisionsabfragen eine Extraspalte links und rechts mit übermittelt
\end{itemize}
\end{frame}


\section{Laufzeitmessungen}
\begin{frame}{Laufzeitmessungen}
\begin{itemize}
	\item 1 Prozess:  285,8s
	\item 2 Prozesse: 74,4s
	\item 4 Prozesse: 23,34s
	\item 8 Prozesse: 11,9s
\end{itemize}
\end{frame}


\section{Leistungsanalyse}
\begin{frame}{Leistungsanalyse}
\begin{itemize}
    \item Die meiste Zeit pro Prozess wird in Kollisionsabfragen benötigt
    \item Senden/empfangen verbringt einen sehr geringen Teil
\end{itemize}
\end{frame}


\section{Skalierbarkeit}
\begin{frame}{Skalierbarkeit}
\begin{itemize}
    \item Super-linearer Speedup bis 8 Prozesse
    \item Danach annährend linearer Speedup
\end{itemize}
\end{frame}
\end{document}