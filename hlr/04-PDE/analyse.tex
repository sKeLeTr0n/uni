\documentclass[a4paper]{scrartcl}
\usepackage[ngerman]{babel}
\usepackage[utf8]{inputenc}
\usepackage[T1]{fontenc}
\usepackage{lmodern}
\usepackage{amssymb}
\usepackage{amsmath}
\usepackage{enumerate}
\usepackage{pgfplots}
\usepackage{scrpage2}\pagestyle{scrheadings}
\usepackage{tikz}
\usetikzlibrary{patterns}

\newcommand{\titleinfo}{Leistungsanalyse}

\title{\titleinfo}
\author{Sönke Kracht, Sven-Hendrik Haase}
\date{\today}
\chead{\titleinfo}
\ohead{\today}
\setheadsepline{1pt}
\setcounter{secnumdepth}{0}
\newcommand{\qed}{\quad \square}

\begin{document}
\maketitle
\notag
\section{Messung 1}

\centerline{
\begin{tikzpicture}[scale=1.5]
  \begin{axis}[
    xlabel=Kerne,
    ylabel=Laufzeit in s
  ]
    \addplot[smooth,color=blue] coordinates {(1, 112.2)
    									     (2, 59.9)
    									     (3, 41.8)
    									     (4, 32.0)
    									     (5, 26.4)
    									     (6, 21.8)
    									     (7, 19.3)
    									     (8, 16.6)
    									     (9, 15.6)
    									     (10, 14.8)
    									     (11, 13.6)
    									     (12, 11.8)};
  \end{axis}
\end{tikzpicture}
}

Wie man sieht, verlaüft der Speed-up in Relation zur Anzahl der Threads 
logarithmisch. Das bedeutet, der Leistungsanstieg wird geringer desto mehr CPUs
im Einsatz sind. Es wird sich kaum lohnen, weitere CPUs einzubinden. Wir sehen
einen deutlichen Leistungsschub beim Schritt von 11 zu 12 Threads im Vergleich
zu den verhergegangenen Schritten. Wir vermuten, dass dies an der
 Hardwarebeschaffenheit des Systems liegt.

\newpage
\section{Messung 2}
\centerline{
\begin{tikzpicture}[scale=1.5]
  \begin{axis}[
    xlabel=Interlines,
    ylabel=Laufzeit in s
  ]
    \addplot[smooth,color=blue] coordinates {(1, 0.005943)
    									     (2, 0.010963)
    									     (4, 0.027519)
    									     (8, 0.061806)
    									     (16, 0.227536)
    									     (32, 0.866702)
    									     (64, 3.972590)
    									     (128, 14.796773)
    									     (256, 55.043998)
    									     (512, 217.514899)
    									     (1024, 953.530896)};
  \end{axis}
\end{tikzpicture}
}

Der Graph zeigt in dem gemessenen Bereich ein lineares Verhaltenswachstum. 
Man kann bei jeder Verdopplung der Interlines eine Vervierfachung der Laufzeit
beobachten. Unter 64 Interlines liegt eine Laufzeit von unter einer Sekunde 
vor, was eine sinnvolle Leistungsbewertung in diesem Bereich unmöglich macht.
In diesem Zusammenhang ist die in der Aufgabenstellung geforderte 
Mindestlaufzeit von 50 Sekunden problematisch, da sonst eine viel zu hohe 
Laufzeit bei 1024 Interlines die Folge ist. Genauer: Damit das Programm bei 1 
Interlines und 12 Threads eine Laufzeit von über 50 Sekunden aufweist, müssten
über 4 Millionen Iterationen berechnet werden. Das hätte bei 1024 Interlines
über 106 Tage Laufzeit zur Folge.

\section{Rückmeldung}

Eine schwammige Formulierung ist uns aufgefallen: "Die schnellste 
Parallelisierung sollte mindestens 50 Sekunden rechnen". Bei wie vielen
Interlines und welche Parametern? Zwar war es Aufgabe, die Parameter entsprechend
zu wählen, aber eine gewisse Vorgabe oder Konkretisierung wäre hier schon 
sinnvoll gewesen.

Ansonsten war die Aufgabe von der Schwierigkeit her angemessen. Wir haben ca.
5 Stunden damit verbracht. 2 Stunden für das Parallelsieren, 1 Stunde für die
Auswertung und 2 Stunden zum Daten sammeln.

\end{document}

